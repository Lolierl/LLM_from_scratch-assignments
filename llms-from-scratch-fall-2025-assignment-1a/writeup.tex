\documentclass{article}
\usepackage{amsmath}
\usepackage[utf8]{inputenc}
\usepackage[ngerman]{babel}
\usepackage{listings}
\usepackage{graphicx}
\usepackage{cite}
\usepackage[UTF8]{ctex}
\begin{document}
\setlength{\parskip}{2em}
\title{LLMs from Scratch: Assignment 1A}
\date{2025.10.12}
\author{计科31 刘思远}
\maketitle
\paragraph{Task 1}
\subparagraph{Problem 1.1}
(a)The minimum and maximum amount of bytes is both $n$. 

(b)The minimum and maximum amount of bits is both $3n$. 

(c)The byte sequence is

[226, 128, 156, 230, 136, 145, 232, 131, 189, 229, 144, 158, 228, 184, 139, 231, 142, 187, 231, 146, 131, 232, 128, 140, 228, 184, 141, 228, 188, 164, 232, 186, 171, 228, 189, 147, 227, 128, 130, 73, 32, 99, 97, 110, 32, 101, 97, 116, 32, 103, 108, 97, 115, 115, 44, 32, 105, 116, 32, 100, 111, 101, 115, 32, 110, 111, 116, 32, 104, 117, 114, 116, 32, 109, 101, 46, 226, 128, 157]

The length of the byte sequence is 79.

\subparagraph{Problem 1.2}
(a)The outputs are as follows:

\verb|Do you like this course? No. Feel boring?\x08\x08\x08\x08\x08\x08\x08\x08\x08\x08\x08\x08\x08\x08\x08\x08\x08 YES!!!!!!!!!!!!!|

\verb|Do you like this course? YES!!!!!!!!!!!!!|

(b)It is the backspace character. It moves the cursor one position backwards, and the next character will override the previous character.

\subparagraph{Problem 1.3}
"It'S 1." is a string satisfying the requirement. It will be pre-tokenized as ["It", "'", "S", "1."], ["It", "'", "S", "1", "."], ["It", "'S", "1", "."], respectively, depending on the pre-tokenization method.

\end{document}
